\section{Podsumowanie}
Projekt można uznać za ukończony. Wszystkie założone funkcjonalności zostały zaprojektowanie i zaimplementowane. Mapper sprostał coraz to nowym wymaganiom stawianym w trakcie trwania semestru. Testy były ważnym narzędziem w trakcie prac nad projektem, nie tylko sprawdzały poprawność wprowadzanych funkcjonalności, ale również nakreślały drogę rozwoju projektu. Dlatego też złożoność testów jest różna, od tych prostszych po te bardziej zaawansowane.

Projekt współpracuje z wieloma formatami, gdzie każdy z nich został sprawdzony przy pomocy przedstawionego zestawu testów. Obsługę wielu formatów można było rozwiązać na dwa główne sposoby:
\paragraph{Pierwszy:} dla każdego formatu można było stworzyć dedykowany mapper, wymagałoby to znacznych zmian w istniejącym kodzie. Tak naprawdę to należałoby rozdzielić istniejącą mechanikę na dwie, obsługę plików \textit{.csv} oraz klasę do zarządzania mapowaniem ogólnie. Wiązałoby się to z bardzo dużym nakładem pracy, natomiast sam projekt znacznie odstawałby od swojej pierworodnej idei. 
\paragraph{Drugi:} można było stworzyć mechanizm konwerterów, współpracujący, z już gotową mechaniką mappera CSV - tak jak to zostało robione. Zastosowane podejście, zdało test, ma ono swoje plusy ale również minusy.
\paragraph{Plusy:}
\begin{itemize}
	\item Stosunkowo mała ilość zmian w istniejącym kodzie.
	\item Bardzo szybki development nowych dostawców formatów.
	\item Rozdzielona odpowiedzialność między konwerterami a mapperem.
	\item Nie ogranicza nas abstrakcja danych w formacie, z perspektywy mappera.
\end{itemize}

\paragraph{Minusy:}
\begin{itemize}
	\item Niska wydajność w porównaniu do dedykowanego mappera konkretnego formatu
	\item Wszelkie niedociągnięcia następujące w trakcie konwersji.
\end{itemize}



